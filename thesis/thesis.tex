\documentclass[titlepage]{article}

\usepackage{textcomp}
\usepackage{xspace}
\usepackage{url}
\usepackage{multirow}
\usepackage{hhline}
\usepackage{pifont}
\usepackage{amsmath, amsfonts , amsthm , mathtools , bbold , float}
\usepackage{tikz}
\usetikzlibrary{arrows,matrix,decorations.pathmorphing,
  decorations.markings, calc, backgrounds}
\usepackage{mathpartir}
\usepackage{microtype}
\usepackage{hyperref}
\usepackage{url}
\DisableLigatures[-]{family=tt*}

\newcommand\myeq{\mathrel{\overset{\makebox[0pt]{\mbox{\normalfont\tiny\sffamily def}}}{=}}}
\newtheorem{theorem}{Theorem}[section]
\newtheorem{definition}{Definition}[section]

%% Configure appearance of system names here
\newcommand{\teletype}[1]{\ensuremath{\mathtt{#1}}}
\newcommand{\systemname}[1]{\teletype{\color{darkgray}#1}\xspace}

%% Systems referred to in the paper
\newcommand{\Agda}{\systemname{Agda}}
\newcommand{\HoTTAgda}{\systemname{HoTT}-\Agda}
\newcommand{\agdaCubical}{\systemname{agda/cubical}}
\newcommand{\Coq}{\systemname{Coq}}
\newcommand{\CubicalAgda}{\systemname{Cubical} \systemname{Agda}}
\newcommand{\cubicaltt}{\systemname{cubicaltt}}
\newcommand{\cooltt}{\systemname{cooltt}}
\newcommand{\Idris}{\systemname{Idris}}
\newcommand{\Haskell}{\systemname{Haskell}}
\newcommand{\Lean}{\systemname{Lean}}
\definecolor{Revolutionary}{RGB}{232,70,68}
\newcommand{\redtt}{\textbf{\teletype{{\color{Revolutionary}red}tt}}}

\newcommand{\eg}{{e.g.}}
\newcommand{\ie}{{i.e.}}
\newcommand{\cf}{{cf.}}

%% Type theory macros
\newcommand{\su}[2]{#1/#2}
\newcommand{\subst}[2]{(\su #1 #2)}
\newcommand{\esubst}[3]{(#1, \su #2 #3)}
\newcommand{\substnop}[2]{{#2}\, / \,{#1}}

%% Agda stuff
\usepackage{agda/latex/agda}
\newcommand{\anum}[1]{\AgdaNumber{#1}}
\newcommand{\symb}[1]{\AgdaSymbol{#1}}
\newcommand{\data}[1]{{\AgdaDatatype{#1}}}
\newcommand{\record}[1]{{\AgdaRecord{#1}}}
\newcommand{\module}[1]{{\AgdaModule{#1}}}
\newcommand{\field}[1]{{\AgdaField{\ensuremath{\mathsf{#1}}}}}
\newcommand{\proj}[1]{\;.\field{#1}}
\newcommand{\func}[1]{\AgdaFunction{#1}}
\newcommand{\prim}[1]{{\AgdaPrimitive{#1}}}
\newcommand{\primty}[1]{{\AgdaPrimitiveType{#1}}}
\newcommand{\II}[0]{{\primty{\ensuremath{\mathbb{I}}}}}
\newcommand{\iz}[0]{\con{i0}}
\newcommand{\io}[0]{\con{i1}}
\newcommand{\hcomp}{\prim{hcomp}}
\newcommand{\hcompcon}{\con{hcomp}}
\newcommand{\var}[1]{{\AgdaBound{#1}}}
\newcommand{\keyw}[1]{{\AgdaKeyword{#1}}}
\newcommand{\con}[1]{{\AgdaInductiveConstructor{\ensuremath{\mathsf{#1}}}}}
\usepackage{catchfilebetweentags}
%% \usepackage{MnSymbol}
\usepackage{newunicodechar}
\newunicodechar{≃}{\ensuremath{\simeq}}
\newunicodechar{≅}{\ensuremath{\cong}}
\newunicodechar{∈}{\ensuremath{\in}}
\newunicodechar{⁻}{\ensuremath{^{-}}}
\newunicodechar{ᴹ}{\ensuremath{_{M}}}
\newunicodechar{ᴺ}{\ensuremath{_{N}}}
\newunicodechar{≡}{\ensuremath{\equiv}}
\newunicodechar{λ}{\ensuremath{\lambda}}
\newunicodechar{⊎}{\ensuremath{\uplus}}
\newunicodechar{∷}{\ensuremath{::}}
\newunicodechar{ℓ}{\ensuremath{\ell}}
\newunicodechar{ᵢ}{\ensuremath{_i}}
\newunicodechar{⟨}{\ensuremath{\langle}}
\newunicodechar{⟩}{\ensuremath{\rangle}}
\newunicodechar{α}{\ensuremath{\alpha}}
\newunicodechar{β}{\ensuremath{\beta}}
\newunicodechar{θ}{\ensuremath{\theta}}
\newunicodechar{φ}{\ensuremath{\varphi}}
\newunicodechar{ψ}{\ensuremath{\psi}}
\newunicodechar{η}{\ensuremath{\eta}}
\newunicodechar{ε}{\ensuremath{\varepsilon}}
\newunicodechar{ι}{\ensuremath{\iota}}
\newunicodechar{Σ}{\ensuremath{\Sigma}}
\newunicodechar{σ}{\ensuremath{\sigma}}
\newunicodechar{∀}{\ensuremath{\forall}}
\newunicodechar{ℕ}{\ensuremath{\mathbb{N}}}
\newunicodechar{→}{\ensuremath{\to}}
\newunicodechar{⊎}{\ensuremath{\uplus}}
\newunicodechar{⋆}{\ensuremath{*}}
\newunicodechar{¬}{\ensuremath{\lnot}}
\newunicodechar{Θ}{\ensuremath{\theta}}
\newunicodechar{∧}{\ensuremath{\wedge}}
\newunicodechar{∨}{\ensuremath{\vee}}
\newunicodechar{∼}{\ensuremath{\sim}}
\newunicodechar{≢}{\ensuremath{\nequiv}}
\newunicodechar{Π}{\ensuremath{\Pi}}
\newunicodechar{ℤ}{\ensuremath{\mathbb{Z}}}
\newunicodechar{∥}{\ensuremath{\parallel}}
\newunicodechar{∣}{\ensuremath{\mid}}
\newunicodechar{ℚ}{\ensuremath{\mathbb{Q}}}
\newunicodechar{₊}{\ensuremath{_+}}
\newunicodechar{∗}{\ensuremath{\ast}}
\newunicodechar{₁}{\ensuremath{_1}}
\newunicodechar{₂}{\ensuremath{_2}}
\newunicodechar{⊥}{\ensuremath{\bot}}
\newunicodechar{⊤}{\ensuremath{\top}}
\newunicodechar{∘}{\ensuremath{\circ}}
\newunicodechar{ρ}{\ensuremath{\rho}}
\newunicodechar{↦}{\ensuremath{\mapsto}}
\newunicodechar{₀}{\ensuremath{_0}}
\newunicodechar{∙}{\ensuremath{\boldsymbol{\cdot}}}
\newunicodechar{Ω}{\ensuremath{\Omega}}
\newunicodechar{§}{\ensuremath{\mathbb{S}}}
\newunicodechar{𝟙}{\ensuremath{\mathbb{1}}}
\newunicodechar{×}{\ensuremath{\times}}

\newunicodechar{⊢}{\ensuremath{\vdash}}
\newunicodechar{ϕ}{\ensuremath{\phi}}
\newunicodechar{Γ}{\ensuremath{\Gamma}}
\newunicodechar{∶}{:}
\newunicodechar{γ}{\ensuremath{\gamma}}
\newunicodechar{ʳ}{\ensuremath{^r}}
\newunicodechar{ˡ}{\ensuremath{^l}}
\newunicodechar{∅}{\ensuremath{\emptyset}}

\newcommand{\winding}[1]{\textsf{winding}({#1})}
\newcommand{\Type}{\func{Type}}
\newcommand{\ptrunc}[1]{\func{∥}\,{#1}\,\func{∥}}
\newcommand{\tyProduct}[2]{{#1}\,\AgdaOperator{\AgdaFunction{×}}\,{#2}}
\newcommand{\tySigma}[3]{\func{Σ[}\,{#1}\,\func{∈}\,{#2}\,\func{]}\,{#3}}
\newcommand{\tyMaybe}[1]{\data{Maybe}\,{#1}}
\newcommand{\tyNat}{\data{ℕ}}
\newcommand{\tyPath}[2]{{#1}\,\func{≡}\,{#2}}
\newcommand{\tyPathP}[4]{\primty{PathP}\,(\symb{λ}\,{#1} \to {#2})\,{#3}\,{#4}}
\newcommand{\tyEquiv}[2]{{#1}\,\func{≃}\,{#2}}
\newcommand{\tySum}[2]{{#1}\,\func{⊎}\,{#2}}
\newcommand{\tyNot}[1]{\func{¬}\,{#1}}
\newcommand{\tyStructEq}[3]{{#1}\,\func{≃[}\,{#2}\,\func{]}\,{#3}}



\bibliographystyle{plainurl}


\begin{document}

%%%%%%%%%%%%%%%%%%%%%%%%%%%%%%%%%%
%            Abstract            %
%%%%%%%%%%%%%%%%%%%%%%%%%%%%%%%%%%

\begin{abstract}
    Abstract
\end{abstract}

\tableofcontents
\thispagestyle{empty}
\newpage
\setcounter{page}{1}


%%%%%%%%%%%%%%%%%%%%%%%%%%%%%%%%%%
%          Introduction          %
%%%%%%%%%%%%%%%%%%%%%%%%%%%%%%%%%%

\section{Introduction}



%%%%%%%%%%%%%%%%%%%%%%%%%%%%%%%%%%
%    The Agda proof assistant    %
%%%%%%%%%%%%%%%%%%%%%%%%%%%%%%%%%%

\section{Agda proof assistant}





%%%%%%%%%%%%%%%%%%%%%%%%%%%%%%%%%%
% Propositional calculus in Agda %
%%%%%%%%%%%%%%%%%%%%%%%%%%%%%%%%%%

\section{Propositional calculus in Agda}
Propositional calculus is a formal system that consists of a set of propositional constants, symbols, inference rules, and axioms. The symbols in propositional calculus represent logical connectives and parenthesis.

The inference rules of propositional calculus define how these symbols can be used. These inference rules specify how to construct well-formed formulas that follow the syntax of the system.

The axioms of propositional calculus are the initial statements or assumptions from which we can derive additional statements using the inference rules.

The semantics of propositional calculus define how the expressions in the system correspond to truth values, typically "true" or "false".


%% Formulas %%

\subsection{Formulas}

\begin{definition}[Language]
    The language $\mathcal{L}$ of propositional calculus consists of
    \begin{itemize}
        \item proposition symbols: $p_0,p_1,\hdots,p_n$,
        \item logical connectives: $\wedge,\vee,\neg,\top,\bot$,
        \item auxiliary symbols: (, ).
    \end{itemize}
\end{definition}

Note that we have omitted the common logical connectives $\rightarrow$ and $\leftrightarrow$. This is becuase we can define them using other connectives, 
\begin{align*}
    \phi \rightarrow \psi &\myeq \neg \phi \vee \psi, \\
    \phi \leftrightarrow \psi &\myeq (\neg \phi \vee \psi) \wedge (\neg \psi \vee \phi),
\end{align*}making them reduntant. It is possible to choose an even smaller set of connectives \cite{vanDalen}, but we choose this as it is convenient.

\begin{definition}[Well formed formula]
    The set of well formed formulas is inductively defined as
    \begin{itemize}
        \item any propositional constant $p_0,p_1,\hdots,p_n$ is a well formed formula,
        \item $\top$ and $\bot$ are well formed formulas,
        \item if $p$ is a well formed formula, then so is
        $$\neg p,$$
        \item if $p_i$ and $p_j$ are well formed formulas, then so are
            $$p_i \wedge p_j \quad \text{and} \quad p_i \vee p_j.$$
    \end{itemize}
    The formula $\top$ should be thought of as the proposition that is always true, and the formula $\bot$ interpreted as the proposition that is always false.
\end{definition}

We can represent the concept of a well formed formula in \Agda as a \Type

\ExecuteMetaData[agda/latex/LindenbaumTarski.tex]{form}


%% Context %%

\subsection{Context}

\begin{definition}[Context]
    A set of sentences in the language $\mathcal{L}$. The set is defined inductively as
    \begin{itemize}
        \item the empty set is a context
        \item if $\Gamma$ is a context, then $\Gamma \cup \{\phi\}$ is also a context, where $\phi$ a formula.
    \end{itemize}
\end{definition}

In \Agda we can define a context \Type

\ExecuteMetaData[agda/latex/LindenbaumTarski.tex]{ctxt}

We also need a way to determine if a given formula is in a given context. 
\begin{definition}[Lookup]
    For all contexts $\Gamma$ and all formulas $\phi$ and $\psi$
    \begin{itemize}
        \item $\phi \in \Gamma \cup \{\phi\}$,
        \item if $\phi \in \Gamma$, then $\phi \in \Gamma \cup \{\psi\}$.
    \end{itemize}
\end{definition}

We represent this as a \Type in \Agda

\ExecuteMetaData[agda/latex/LindenbaumTarski.tex]{lookup}


%% Inference rules %%

\subsection{Inference rules}

\subsubsection{Conjunction}
\subsubsection{Disjunction}
\subsubsection{Negation}
\subsubsection{Rules for $\top$ and $\bot$}
\subsubsection{Axiom}
\subsubsection{Law of excluded middle}
\subsubsection{Structural rules}

\ExecuteMetaData[agda/latex/LindenbaumTarski.tex]{rules}


%% Properties of propositional calc %%

\subsection{Properties of a propositional calculus}





%%%%%%%%%%%%%%%%%%%%%%%%%%%%%%%%%%
%   Lindenbaum Tarski algebra    %
%%%%%%%%%%%%%%%%%%%%%%%%%%%%%%%%%%

\section{Lindenbaum-Tarski algebra}
[What is LT?]

\subsection{Representing Lindenbaum Tarski algebra in Agda}
\subsection{Proof that the Lindenbaum Tarski algebra is Boolean}
\subsection{Soundness}


\bibliography{refs}

\end{document}